\documentclass[12pt,a4paper]{article}
\usepackage[utf8]{inputenc}
\usepackage[spanish]{babel}
\usepackage{graphicx}
\usepackage{hyperref}
\usepackage{enumitem}

\title{La Obsolescencia Programada:\\ 
Reflexiones sobre Consumo y Sostenibilidad}
\author{Documental: ``The Light Bulb Conspiracy''}
\date{}

\begin{document}

\maketitle

\section*{Introducción}
Este documento contiene preguntas y puntos de discusión basados en el documental ``The Light Bulb Conspiracy'', que explora cómo los productos están diseñados para fallar y cómo esto afecta a nuestro planeta y sociedad.

\section{Preguntas de Reflexión}

\subsection*{Sobre la Obsolescencia Programada}
\begin{enumerate}
    \item ¿Qué es la obsolescencia programada? ¿Puedes dar ejemplos de productos que crees que están diseñados para fallar?
    \item ¿Por qué crees que las empresas diseñan productos para que fallen después de un tiempo determinado?
    \item ¿Cómo afecta la obsolescencia programada a tu vida diaria? ¿Qué productos has tenido que reemplazar recientemente?
\end{enumerate}

\subsection*{Impacto Ambiental}
\begin{enumerate}
    \item ¿Qué consecuencias tiene para el medio ambiente el constante reemplazo de productos?
    \item ¿Dónde terminan todos los productos que desechamos? ¿Qué problemas crea esto?
    \item ¿Qué soluciones podrías proponer para reducir el impacto ambiental de la obsolescencia programada?
\end{enumerate}

\subsection*{Consumo y Sociedad}
\begin{enumerate}
    \item ¿Cómo crees que la publicidad influye en nuestro deseo de comprar nuevos productos?
    \item ¿Qué papel juegan las redes sociales en fomentar el consumo de nuevos productos?
    \item ¿Crees que es posible vivir sin caer en el ciclo de comprar-tirar-comprar?
\end{enumerate}

\section{Datos Interesantes del Documental}
\begin{itemize}
    \item La bombilla de Livermore: ¿Sabías que existe una bombilla que lleva funcionando más de 100 años?
    \item El caso de las impresoras: ¿Por qué las impresoras dejan de funcionar después de cierto número de impresiones?
    \item El diseño de los productos: ¿Cómo se diseñan los productos para que fallen en un momento determinado?
\end{itemize}

\section{Actividades de Debate}
\begin{enumerate}
    \item \textbf{Debate en grupo:} ¿Debería ser ilegal la obsolescencia programada? Divide la clase en dos grupos y debate los pros y contras.
    
    \item \textbf{Proyecto práctico:} Investiga la vida útil de diferentes productos en tu casa. ¿Cuánto tiempo duran? ¿Por qué fallan?
    
    \item \textbf{Role-play:} Simula una reunión entre consumidores y fabricantes para discutir la durabilidad de los productos.
\end{enumerate}

\section{Reflexiones Finales}
\begin{itemize}
    \item ¿Qué cambios podrías hacer en tu vida diaria para reducir el consumo?
    \item ¿Cómo podrías influir en tu familia y amigos para que sean más conscientes del consumo?
    \item ¿Qué responsabilidad tienen los gobiernos y las empresas en este problema?
\end{itemize}

\section*{Citas Destacadas del Documental}
\begin{itemize}
    \item \textbf{``Anyone who thinks that infinite growth is consistent with a finite planet is either crazy, or an economist.''} \\
    \textit{``Cualquiera que piense que el crecimiento infinito es compatible con un planeta finito está loco, o es un economista.''}
    
    \item \textbf{``The problem is that now we've all become economists.''} \\
    \textit{``El problema es que ahora todos nos hemos convertido en economistas.''}
    
    \item \textbf{``With this Growth Society we are sitting in a racing car that no longer has a driver, is running at full speed and will end up crashing into a wall or driving off a cliff.''} \\
    \textit{``Con esta Sociedad del Crecimiento estamos sentados en un coche de carreras que ya no tiene conductor, va a toda velocidad y terminará chocando contra un muro o cayendo por un precipicio.''}
    
    \item \textbf{``The three crucial factors are advertising, planned obsolescence and credit.''} \\
    \textit{``Los tres factores cruciales son la publicidad, la obsolescencia programada y el crédito.''}
    
    \item \textbf{``Growth Society's logic is not only to grow to meet demand but to grow for the sake of growth.''} \\
    \textit{``La lógica de la Sociedad del Crecimiento no es solo crecer para satisfacer la demanda, sino crecer por el simple hecho de crecer.''}
\end{itemize}

\section*{Recursos Adicionales}
\begin{itemize}
    \item Documental completo: ``The Light Bulb Conspiracy''
    \item Sitios web sobre consumo responsable
    \item Organizaciones que promueven la sostenibilidad
\end{itemize}

\end{document}
